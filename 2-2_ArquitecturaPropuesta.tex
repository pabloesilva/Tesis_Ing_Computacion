\cleardoublepage                                  % 
\chapter{Arquitectura del sistema}

El Sistema de Gestión Consumos (SGC) propuesto plantea un esquema distribuido para la gestión de las cargas en una la microrred, enfocado en
equilibrar la demanda de consumo con la potencia disponible en tiempo real. El SGC pretende garantizar un uso eficiente
de la energía, priorizando el abastecimiento de las cargas críticas de la microrred.
La Figura~\ref{img:arquitectura} muestra los elementos que componen al SGC propuesto. El Agente de Carga (AC) es un
dispositivo construido en base a un microcontrolador, que es responsable de establecer el vínculo entre los recursos
energéticos de la microrred y los Nodos de Consumo (NC).
El AC establece comunicación con los demás agentes de la microrred mediante un bus de comunicación CAN, a fin de
obtener el valor de potencia disponible, cuyo valor depende directamente de las condiciones de generación y/o
almacenamiento que posee la microrred. El AC también realiza medición del voltaje eficaz en la línea de corriente
alterna proveniente de la microrred; con este dato de tensión y el valor de potencia disponible, calcula la corriente eficaz
total que puede utilizar el conjunto de cargas e informa este valor periódicamente a los NC. Estos últimos, constituyen
dispositivos de control (basados en microcontroladores) que están asociados a diferentes cargas eléctricas (por ej.
electrobombas, electrodomésticos, iluminación, entre otros).
Cada NC dispone de una prioridad asignada, que es definida según la necesidad especifica del usuario. La comunicación
entre los nodos se realiza de manera multidireccional, permitiendo que cada uno, a partir de mensajes de todos los
demás, evalúe su condición operativa en relación con el valor de la corriente total disponible en la microrred, su nivel de
consumo y prioridad frente a otros nodos. Con esa información, cada nodo toma la decisión de conexión, desconexión o reconexión de la carga que tiene asociada.

\begin{figure}[hbt!]
    \centering
    \includegraphics[width=0.7\textwidth]{imagenes/arquitectura.png}
    \caption{Arquitectura para el sistema propuesto}
    \label{img:arquitectura}
\end{figure}

\section{Agente de Carga (AC)}

El AC es el nexo entre la microrred y los NC, recibiendo la potencia disponible desde el sistema multiagente de supervisión de la
microrred. A través de esta potencia y la medición local de la tensión eficaz de corriente alterna, el AC obtiene la corriente
total disponible para el consumo en las cargas. Esta información de corriente disponible es reportada periódicamente a todos los NC quienes toman este valor como una restricción para su funcionamiento, de modo que éstos puedan ejecutar la lógica distribuida de conexión y desconexión de cargas, según las prioridades establecidas para cada una de ellas.
Para cumplir con esto, el AC realiza las siguientes tareas esenciales:
\begin{itemize}
    \item Comunicación vía bus CAN con el sistema de supervisión de la microrred para obtener la potencia disponible y reportar el consumo.
    \item Medición de la tensión eficaz de la línea para el cálculo de la corriente disponible.
    \item Envío de datos de la microrred (potencia, tensión, corriente) y de cada NC mediante UART para su posterior almacenamiento y visualización en una interfaz web.
\end{itemize}

Esta estrategia facilita el balanceo de la demanda frente a la oferta y prioriza el abastecimiento de las cargas críticas cuando la energía es limitada. En la Figura~\ref{img:esq_ac} se presenta el diagrama esquemático del AC diseñado para este sistema, donde se observan los bloques funcionales que permiten cumplir con las tareas mencionadas.

\begin{figure}[hbt!]
    \centering
    \includegraphics[width=0.5\textwidth]{imagenes/esq_ac.png}
    \caption{Diagrama esquemático del Agente de Carga (AC)}
    \label{img:esq_ac}
\end{figure}

\section{Nodo de Consumo (NC)}
Los Nodos de Consumo son las unidades responsables de decidir, de manera autónoma, si una carga eléctrica debe permanecer conectada o desconectada. Para ello, cada NC participa en un esquema de control distribuido donde intercambia información con los demás nodos y con el Agente de Carga (AC). Esta comunicación le permite conocer dos valores esenciales: la disponibilidad total de corriente para la microrred, que es informada periódicamente por el AC, y el consumo agregado de las cargas, estimado mediante el intercambio de mensajes entre los propios nodos.
A partir de estos datos, cada NC evalúa su situación considerando la prioridad que tiene asignada. Las cargas críticas se mantienen siempre conectadas, mientras que las cargas no críticas pueden ser desconectadas si la demanda total supera la disponibilidad. Esta decisión no depende de un controlador central, sino que surge del propio nodo aplicando un algoritmo de consenso y priorización que asegura que la demanda global se mantenga por debajo del límite informado por el AC.
Cada NC cuenta con una estructura, mostrada en la Figura~\ref{img:esq_nc}, para llevar a cabo la logica de control distribuido y realizar la actuación sobre la carga asociada.

\begin{figure}[H]
    \centering
    \includegraphics[width=0.5\textwidth]{imagenes/esq_nc.png}
    \caption{Diagrama esquemático del Nodo de Consumo (NC)}
    \label{img:esq_nc}
\end{figure}

\subsection{Lógica del Algoritmo de priorización de las Cargas}
\begin{figure}[hbt!]
    \centering
    \includegraphics[width=0.6\textwidth]{imagenes/Diagrama-Flujo.png}
    \caption{Diagrama de Flujo del algoritmo de priorización de las cargas en el NC}
    \label{img:Diagrama-Flujo}
\end{figure}

\section{Visualización de datos de manera remota}
\begin{figure}[hbt!]
    \centering
    \includegraphics[width=0.7\textwidth]{imagenes/Plataforma-IoT.png}
    \caption{Plataforma-IoT Propuesta}
    \label{img:plataforma-iot-prop}
\end{figure}