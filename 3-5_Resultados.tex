\cleardoublepage                                  % 
\chapter{Resultados}

Los resultados deben presentarse de forma clara y concisa, evitando la tentación de presentarlos en más de un preceder cada subsección de resultados con un párrafo de resumen, pero evitando la duplicación. Las tablas y figuras deben ser lo más sencillas posible y debe evitarse el uso de gráficos muy complicados o combinaciones de colores oscuros: ¡el examinador no se lo agradecerá! La tabla o figura no debe repetir información del texto principal, sino aumentarla. Todas las tablas y figuras deben tener un título que explique de forma concisa lo que se presenta. Si se utilizan abreviaturas, es importante explicarlas detalladamente. 

Las tablas en las que se citan los valores p deben indicar el valor p real, en lugar de p<0,05; p<0,01, p<0,001. Con el uso generalizado de paquetes estadísticos informatizados, el valor p real puede encontrarse con relativa facilidad.

\section{Comunicaciones}

\section{Plataforma IoT}

\section{Control prioritario de cargas}