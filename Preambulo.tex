%!TEX root = 1_Tesis.tex

\RequirePackage{etex}

\documentclass[
titlepage,
oneside,
openright,
onecolumn,
a4paper,
12pt]{book}

% Text encoding to take for data input stream. I.e. the text we write in.
% How to encode fonts.
%\usepackage[T1]{fontenc}
%usepackage{ucs}              %% Unicode support
%\usepackage[T1]{fontenc}
\usepackage[utf8]{inputenc}
\usepackage{amsmath}
\usepackage{epigraph}

% Para mostrar el icono ORCID
\usepackage{academicons}
\usepackage{fontawesome5}
\usepackage{etoolbox}

% Leave the smallest margins on A4 paper.
\usepackage{a4wide}

% Configuracion de los margenes de la pagina
% Ver: https://es.sharelatex.com/learn/Page_size_and_margins
\usepackage{geometry}
\geometry{
 a4paper,
 left=3.5cm, right=2.0cm,
 top=2.5cm, bottom=2.5cm}

% Separacion entre párrafos
%\setlength\parskip{\medskipamount}
\setlength\parskip{\bigskipamount}

% Separacion en las listas
\usepackage{enumitem}
\setlist{noitemsep} % Uncomment this if you want it as a global setting

% Sangría en la primera línea
\setlength\parindent{50pt}

% Para importar archivos pdf (la portada)
\usepackage{pdfpages} 
\usepackage{pdflscape}

% Para comentar parrafos
\usepackage{comment}

% Tamaño de la fuente del pie de imagen
\usepackage{caption}
\captionsetup{font=footnotesize}
\captionsetup[figure]{name=Figura} % se agrega para cambiar "Ímagen" por "Figura"
% Fancy LaTeX chapter styles
% Options: Sonny, Lenny, Glenn, Conny, Rejne, Bjarne, Bjornstrup
\usepackage[Sonny]{fncychap}

\ChNumVar{\fontsize{76}{80}\usefont{OT1}{pzc}{m}{n}\selectfont}
\ChTitleVar{\raggedleft\huge\sffamily\bfseries}  
\ChNameVar{\bfseries\Large\sf}
%\ChNumVar{\Huge}
%\ChTitleVar{\bfseries\Large\rm}
\ChRuleWidth{1pt}
%\ChNameUpperCase
%\ChTitleUpperCase


% Para insertar hipervinculos  %% SI se actvia, no muestra los indices
\usepackage[pageanchor]{hyperref}
%\usepackage{url} Reduntante su see usa hyperref
%\usepackage[anythingbreaks]{breakurl} % Para cortar las URL largas

\hypersetup{
    colorlinks=true,
    linkcolor=blue,
    filecolor=magenta,      
    urlcolor=cyan,
    citecolor=blue,
    pdfpagemode=FullScreen,
    %backref=page,
}
\urlstyle{same}


\usepackage{cleveref}
\usepackage{chngcntr}

% Ver tipografias en: The LaTeX Font Catalogue http://www.tug.dk/FontCatalogue/ 
%% \usepackage{palatino} 
% %\usepackage{tgbonum}
%\usepackage{antpolt}
% %\usepackage{librebaskerville}
% \usepackage{inconsolata}
\usepackage{antpolt}

\usepackage[tight,spanish]{minitoc}

\usepackage[spanish, es-tabla, activeacute]{babel}
\addto\captionsspanish{% se agrega para cambiar "Índice de Imágenes" por "Índice de Figuras"
  \renewcommand{\listfigurename}{Índice de Figuras}%
}

\usepackage{fvextra}
\usepackage{csquotes}

\usepackage[
    backend=biber,
    style=ieee,
    sorting=none
]{biblatex}
\addbibresource{Bibliografia.bib} 

% We need double spacing for the draft, at least. Also required by Univ. PTA.
\usepackage{setspace}
\onehalfspacing
%\doublespacing


% Glosario de términos
%\usepackage{acronym}
%\usepackage[acronym,nopostdot,nogroupskip]{glossaries}
%\usepackage{glossaries}
\usepackage[automake,acronym,nopostdot,nogroupskip]{glossaries-extra}
%\usepackage[version=3]{acro} %list of abbreviations
%\usepackage{acro}[=v3]
%%% Acrónimos %%%
%%%%%%%%%%%%%%%%%%%%

\newacronym{URL}{URL}
{Localizador Uniforme de Recursos (Uniform Resource Locator), una dirección web que especifica la ubicación de un recurso en Internet y el mecanismo para acceder a él}

\newacronym{HTTP}{HTTP}
{Protocolo de Transferencia de Hipertexto (Hypertext Transfer Protocol), utilizado para la comunicación y transferencia de datos en la World Wide Web}

\newacronym{WWW}{WWW}
{World Wide Web, sistema global de información basado en hipertexto que permite acceder y navegar por múltiples recursos en Internet}

\newacronym{P2P}{P2P}
{Red “peer-to-peer” (par a par), un modelo de intercambio directo de recursos entre usuarios sin necesidad de un servidor central}


%%% Definiciones %%%
%%%%%%%%%%%%%%%%%%%%

\newglossaryentry{metadatos}
{
    name = metadatos,
    description = {Datos estructurados que describen, explican y permiten localizar o gestionar recursos de información}
}

\newglossaryentry{compilador}
{
    name = compilador,
    description = {Programa que transforma código fuente escrito por un programador en instrucciones ejecutables por una computadora}
}

\newglossaryentry{plataforma}
{
    name = plataforma,
    description = {Infraestructura tecnológica y sociocultural que integra hardware, software, algoritmos y políticas, mediando prácticas sociales y producciones de contenido}
}

\newglossaryentry{multiplataforma}
{
    name = multiplataforma,
    description = {Software o tecnología capaz de ejecutarse e interoperar en múltiples sistemas operativos o entornos digitales}
}

\newglossaryentry{online}
{
    name = online,
    description = {Estado de conexión a una red digital, generalmente Internet, o actividad realizada a través de ella}
}

\newglossaryentry{twitter}
{
    name = Twitter,
    description = {Servicio de microblogging para la publicación breve y dinámica de mensajes}
}

\newglossaryentry{LaTeX}
{
    name = LaTeX,
    description = {Sistema de preparación de documentos basado en TeX, desarrollado por Leslie Lamport, ampliamente utilizado en el ámbito académico y científico}
}

\makeglossaries
%\makeglossaries[main,acronym]
%\makenoidxglossaries



% Nomenclarura (para las variables)
\usepackage[refpage,noprefix]{nomencl}
\makenomenclature
\usepackage{morewrites}
% see https://texfaq.org/FAQ-noroom
%\usepackage{etex} 
%\reserveinserts{28}

% Apendices
\usepackage[toc,page]{appendix}


\usepackage{multicol}         
\usepackage{ccicons} % Iconos Creative Commons

% Para listar código
\usepackage{color}
\definecolor{bluekeywords}{rgb}{0.13,0.13,1}
\definecolor{greencomments}{rgb}{0,0.5,0}
\definecolor{redstrings}{rgb}{0.9,0,0}

\usepackage{listings}
\lstset{
%language=XML,
showspaces=false,
showtabs=false,
breaklines=true,
showstringspaces=false,
breakatwhitespace=true,
escapeinside={(*@}{@*)},
commentstyle=\color{greencomments},
keywordstyle=\color{bluekeywords}\bfseries,
stringstyle=\color{redstrings},
basicstyle=\scriptsize\ttfamily,
columns=fullflexible
}

\lstdefinelanguage{OSM}
{
	morekeywords={node, id, lat, lon, user, uid, visible, version, changeset, timestamp, tag},
    sensitive=false,
}

% para las tablas
\usepackage{xcolor}
\definecolor{LightGray}{gray}{0.9}
\usepackage{lscape}
\usepackage{booktabs}
\usepackage{multirow}
\usepackage{longtable}
\usepackage{colortbl} % Paquete para agregar color a las celdas

\usepackage{graphicx,amsfonts,psfrag,fancyhdr,layout,appendix}
\usepackage{subfigure}

\usepackage[nottoc,notlot,notlof]{tocbibind}   
\usepackage{xpatch}


\usepackage[subfigure]{tocloft}
% Para agregar puntos en los niveles de la tabla de contenidos
\renewcommand{\cftpartleader}{\cftdotfill{\cftdotsep}} % for parts
\renewcommand{\cftchapleader}{\cftdotfill{\cftdotsep}} % for chapters
\renewcommand{\cftsecleader}{\cftdotfill{\cftdotsep}} % for sections

% Añadir las ecuaciones al índice
%\usepackage{etoolbox}
%\AtBeginDocument{\listofequations}

% para poner titulos a las ecuaciones
%\usepackage{capt-of} % Para usar \captionof

\usepackage{makeidx} 
\usepackage{float}
\usepackage{textcomp}

% Para listar los algoritmos
%\usepackage{algorithm2e}
\usepackage{algorithm}
\usepackage{algpseudocode}
\usepackage{listings}
\usepackage{minted}
\usemintedstyle{colorful}

\usepackage{lettrine}
\usepackage{xstring}

\widowpenalty=100000
\clubpenalty=100000
\raggedbottom
%%%% \hyphenpenalty=10000 (almost) prevents hyphenation
\hyphenpenalty=9000
\tolerance=100

% Para queue aparezcan hasta las subsubsecciones en el índice
%\usepackage{etoc}
%\setcounter{secnumdepth}{4}
%\setcounter{tocdepth}{4}

\usepackage{eqparbox}

\usepackage{tocloft} 
\cftsetindents{figure}{1.5em}{3.2em} % Indentación del índice de figuras (se puede variar)
\cftsetindents{chapter}{0em}{3.2em}
\cftsetindents{section}{1.5em}{3.2em}
\cftsetindents{subsection}{3.0em}{3.2em}
%%%%%%%%%%%%%%%%%%%%%%%%%%%%%%%%%%%%%%%%%%%%%%%%%%%%%
%%Cajas para las ecuaciones
\usepackage{varwidth}
%\usepackage{etoolbox}
\usepackage[most,many,skins,breakable]{tcolorbox}
\usepackage{tikz,lipsum,lmodern}

\newtcolorbox{caja2}[2][]{enhanced, interior hidden, 
coltitle=black, fonttitle=\bfseries\headingfont\large,
attach boxed title to top left={yshift=-2.5mm},
boxed title style={empty, size=small, top=1mm, bottom=2pt},
%boxed title=0.5\linewidth,
frame code={
\path (title.east|-frame.north) coordinate (aux);
\path[draw=blue, fill=blue!5, line width=0.5mm, rounded corners]
(frame.west) |- ([xshift=-2.5mm]title.north east) to[out=0, in=180] ([xshift=7.5mm]aux)-|(frame.east)|-(frame.south)-|cycle;  
},
title={#2},#1}
%%%%%%%%%%%%%%%%%%%%%%%%%%%%%%%%%%%%%%%%%%%%%

\newtcolorbox{caja}[2][]{empty, 
coltitle=black,
fonttitle=\bfseries\sffamily,
attach boxed title to top left={yshift=-2.5mm},
boxed title style={empty, size=small, top=1mm, bottom=0pt},
varwidth boxed title=0.5\linewidth,
frame code={
\path (title.east|-frame.north) coordinate (aux);
\path[draw=black, line width=0.5mm, rounded corners]
(frame.west) |- ([xshift=-2.5mm]title.north east) to[out=0, in=180] ([xshift=7.5mm]aux)-|(frame.east)|-(frame.south)-|cycle;  
},
title={#2},#1}


%%% Definiciones:
\usepackage{tcolorbox}
\tcbuselibrary{theorems}
\newtcbtheorem[number within=section]{definicion}{Definición}%
{colback=black!5!white,colframe=black!75!black,fonttitle=\bfseries}{th}
%%%%%%%%%%%%%%%%%%%%%%%%%%%%%%%%%%


\usepackage{xspace}
\usepackage{morewrites}
