\cleardoublepage
\chapter{Prólogo}
%\addcontentsline{toc}{chapter}{Prólogo}

La transición hacia un modelo energético más sostenible y descentralizado ha dejado de ser una proyección futura para convertirse en una necesidad presente. En este contexto, las microrredes eléctricas emergen como la solución tecnológica por excelencia para dotar de autonomía y resiliencia a comunidades donde el tendido eléctrico convencional es deficiente o inexistente. Sin embargo, la implementación exitosa de estas redes no depende únicamente de la capacidad de generación, sino de la inteligencia con la que se administra el recurso finito de la energía.

El presente trabajo de tesis aborda, desde una perspectiva de ingeniería aplicada, el desafío crítico de equilibrar la oferta y la demanda en sistemas aislados de pequeña escala. A diferencia de los enfoques tradicionales centralizados, aquí se explora un paradigma distribuido, donde la toma de decisiones recae en dispositivos inteligentes dispersos en la red, capaces de cooperar para garantizar la continuidad del servicio.

A lo largo de estas páginas, el lector encontrará una propuesta integral que abarca desde el diseño del hardware electrónico y la programación de firmware de tiempo real, hasta la implementación de protocolos de comunicación robustos y plataformas de supervisión IoT. Este documento ha sido estructurado para guiar al lector desde los fundamentos teóricos del control jerárquico hasta la validación experimental de un prototipo funcional, ofreciendo una visión técnica detallada de cómo la electrónica moderna puede mejorar la calidad de vida en entornos rurales.