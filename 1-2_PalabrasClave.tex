\cleardoublepage                              
\chapter{Palabras Clave}


\textbf{\textit{ Generación Distribuida; Microrredes;
Energías Renovables; Escuelas de la Familia Agrícola (EFA); Gestión de Energía; Control Jerárquico;}}


% \section{Algunas instrucciones útiles}\label{Instrucciones}

% Para separar los párrafos se deja una línea en blanco entre ellos, lo que hace que se indente automáticamente la primera línea del siguiente párrafo.

% \subsection{Formato de texto}

% \textbf{Texto en Negrita}, \textit{Texto en Cursiva}, \underline{Texto Subrayado}, \\
% \textbf{\textit{\underline{Texto en Negrita, Cursiva y Subrayado}}}.

% \subsection{Citas}

% Para citar una referencia incluida en el archivo \texttt{Bibliografia.bib}, se utiliza el comando \verb|\cite|.  
% Por ejemplo, Castells aparece citado como \cite{Castells2001}.  
% Otro libro del mismo autor se referencia como \cite{Castells2009}.

% Estas instrucciones pueden verse con más detalle en \cite{Brys2019}.

% Para poner una nota al pie que no sea una cita bibliográfica, se escribe así:%
% \footnote{Esto es una nota al pie ajena al sistema de referencias.}

% \subsubsection{Hipervínculos}

% Para colocar un enlace directo a una URL, se utiliza: \url{http://www.unam.edu.ar}.  
% Si se desea texto clickeable, se escribe: \href{http://www.hycs.unam.edu.ar}{Sitio de la FHyCS-UNaM}.

% Una imagen se agrega colocando primero el archivo en la carpeta \texttt{imagenes}.  
% Si es necesario referenciarla, por ejemplo figura \ref{img:NombreDeLaImagen}:

% \begin{figure}[hbt!]
% \centering
% \includegraphics[scale=0.50]{imagenes/logo-unam.png}
% \caption{Descripción de la imagen que se muestra}
% \label{img:NombreDeLaImagen}
% \end{figure}

% \subsection{Referencias}\label{Referencias}

% También es posible referenciar capítulos u otros elementos con etiquetas \verb|\label|.  
% Por ejemplo, este tema se retoma en \cref{Ayudas}, cuyo título es \nameref{Ayudas},  
% y específicamente en el subtítulo \cref{Referencias} en la página \pageref{Referencias}.

% \subsection*{Listas}

% Una lista con viñetas:
% \begin{itemize}
%     \item Primer elemento,
%     \item Segundo elemento.
% \end{itemize}

% Una lista numerada:
% \begin{enumerate}
%     \item Primer elemento,
%     \item Segundo elemento.
% \end{enumerate}

% Las comillas en español se escriben así: “ ”.  
% Ejemplo de subíndice: \[ x_{1}+x_{2}=0 \]  
% Ejemplo de superíndice: \[ x^{2}+y^{2}=r^{2} \]

% Aquí hay un modelo de citas bibliográficas con estilo IEEE:

% Cita estándar: \cite{Foucault1988}.  
% El libro que dio origen a este proyecto: \cite{Brys2019}.  

% Otras referencias relevantes del trabajo:  
% \cite{Bajtin2005}, \cite{Bourdieu2005}, \cite{Castells2001},  
% \cite{Castells2006}, \cite{Castells2009}, \cite{Dreyfus2001},  
% \cite{Veron1993}, \cite{Veron2001}, \cite{Wolton2007}.  

% Referencia a imágenes:  
% Como se muestra en la figura \ref{img:logo-unam}, en la figura \ref{img:logo-fce}  
% y en la figura \ref{img:logo-fhycs}.

% Definiciones del glosario:  
% \gls{LaTeX}, \gls{metadatos}, \gls{compilador}, \gls{nube}, \gls{conectividad},
% \gls{plataforma}, \gls{multiplataforma}, \gls{online}, \gls{twitter}.  

% Acrónimos: \glsxtrshort{URL}, \glsxtrshort{HTTP}, \glsxtrshort{WWW}, \glsxtrshort{P2P}.

% \begin{figure}[hbt!]
% \centering
% \includegraphics[scale=0.20]{imagenes/latex-logo.png}
% \caption{Isologo de LaTeX}
% \label{img:latex-logo}
% \end{figure}

% Los isologos de la Universidad Nacional de Misiones, de la Facultad de Humanidades y Ciencias Sociales y de la Facultad de Ciencias Económicas están publicados bajo una licencia Creative Commons \cc \ccby \ccsa.  

% \begin{figure}[hbt!]
% \centering
% \includegraphics[scale=0.30]{imagenes/logo-unam.png}
% \caption{Isologo de la Universidad Nacional de Misiones}
% \label{img:logo-unam}
% \end{figure}

% \begin{figure}[hbt!]
% \centering
% \includegraphics[scale=0.20]{imagenes/logo-fce.jpg}
% \caption{Isologo de la Facultad de Ciencias Económicas}
% \label{img:logo-fce}
% \end{figure}

% \begin{figure}[hbt!]
% \centering
% \includegraphics[scale=0.30]{imagenes/logo-fhycs.png}
% \caption{Logotipo de la Facultad de Humanidades y Ciencias Sociales}
% \label{img:logo-fhycs}
% \end{figure}

% \noindent Autor: Sonia Lutjohann \\
% \noindent Diseñado con Blender 3D
