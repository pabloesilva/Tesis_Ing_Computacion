\cleardoublepage
\chapter{Introducción}
\label{cap:Introduccion}

\epigraph{``La ciencia es el padre del conocimiento, pero las opiniones son las que engendran la ignorancia."}{\textit{Hipócrates}}

A nivel mundial la producción de energía eléctrica está basada predominantemente en un modelo centralizado, donde grandes centrales de generación, interconectadas entre sí, abastecen a los centros de consumo. Estas centrales mayormente utilizan para la generación combustibles fósiles, energía hidráulica o energía nuclear y suelen ubicarse lejos de los consumidores. La gran distancia entre los puntos de generación y los usuarios finales implica que la energía eléctrica deba ser transportada y distribuida mediante extensas redes, propiciando dos desventajas principales: una mayor vulnerabilidad a las interrupciones generalizadas del suministro y pérdidas de energía significativas en la transmisión \cite{Mehigan2018}. Por otra parte, a esto se agrega que las instalaciones de producción a gran escala presentan un impacto ambiental considerable, ya que la construcción y operación de estas alteran profundamente los ecosistemas en los que son instalados \cite{Olivera2010}.

Actualmente el modelo de generación centralizado está siendo modificado debido al incremento de sistemas de generación que son conectados a la red de distribución, tanto por clientes industriales como residenciales. En los últimos años ha tomado relevancia este tipo de generación distribuida, propiciando una solución a los inconvenientes surgidos con el esquema de generación centralizada.

La Generación Distribuida (GD) consiste en la producción de energía eléctrica mediante pequeñas fuentes de generación ubicadas cerca de los puntos de consumo, en contraposición a lo que sucede en el modelo tradicional centralizado \cite{Shaukat2023}. En este esquema de generación se utiliza una amplia gama de tecnologías, generalmente (aunque no siempre) bajas en emisiones de carbono (por ej. solar, eólica, biomasa) o más eficientes (por ej. cogeneración de calor y electricidad), que proporcionan beneficios asociados al ahorro en la transmisión y distribución de la energía (reducción de pérdidas), eliminación de costosas infraestructuras, incremento en la diversificación de fuentes de energía, adaptación a las demandas (usando generadores modulares) y mejoras al acceso a la energía en áreas remotas o rurales, entre otras \cite{Allan2015}. La GD permite una producción de energía eléctrica más accesible, eficiente, fiable y con menor impacto al medioambiente que en el caso del esquema de generación centralizado.

Con el fin de integrar diferentes recursos energéticos distribuidos a las redes eléctricas existentes, en el año 2002 surge el concepto de Microrred \cite{Lasseter2002}. El mismo es definido como aquellos sistemas integrados, que combinan elementos tales como generadores, dispositivos de almacenamiento y cargas eléctricas ubicados dentro de un área delimitada, con la capacidad de operar tanto de forma autónoma como conectadas a la red de distribución principal \cite{Olulope2022}. El término Microrred hace referencia a un pequeño número de recursos energéticos (fuentes y almacenadores de energía) que puede actuar como una entidad única a nivel de red y proporcionar un modo de funcionamiento desconectado (autónomo) o conectado a la red, brindando así el suministro de energía permanente a las cargas eléctricas locales \cite{Uddin2023}. En el modo de operación autónomo puede aprovecharse los recursos energéticos locales (fuentes y almacenamiento de energía), satisfaciendo la carga de la microrred; mientras que, en el modo conectado a la red, es habilitado el intercambio de energía con la misma. La importación de energía a la microrred puede darse cuando la disponibilidad de los recursos energéticos es insuficiente, mientras que la microrred podrá exportar energía a la red eléctrica de distribución cuando hay excedentes en la generación de energía eléctrica. Estos modos de operación permiten la adaptación de las microrredes a diferentes condiciones requeridas por sus elementos, mejorando la eficiencia, confiabilidad y resiliencia del sistema eléctrico que abastece a las cargas \cite{Zidane2025}. Esto contribuye a los beneficios propios de los sistemas de GD mencionados, haciendo que la red eléctrica ya no sea un sistema unidireccional de transferencia de energía.

Considerando el bus que permite la transferencia de energía entre los componentes de las microrredes (generadores, sistemas de almacenamiento y cargas), estas pueden clasificarse en tres tipos fundamentales \cite{Kim2023,Lei2025}:

Microrredes de corriente alterna (M-CA): poseen un bus o barra de CA al que se conectan generación, almacenamiento y cargas. Estos sistemas presentan una integración sencilla a la red eléctrica convencional; utilizan tecnologías maduras, especialmente en convertidores y transformadores; pero se requieren conversiones extras cuando las fuentes y/o cargas son de CC (ej. paneles fotovoltaicos, baterías, electrónica de potencia, entre otras), lo que reduce la eficiencia.

Microrredes de corriente continua (M-CC): se estructuran alrededor de un bus de CC. Presentan una eficiencia más alta al evitar múltiples conversiones (paneles fotovoltaicos y baterías pueden conectarse directamente); son adecuadas para cargas modernas que ya trabajan en CC (iluminación LED, vehículos eléctricos, entre otros); no presentan los inconvenientes de sincronización y estabilidad típicos de CA. Pero aún falta estandarización en niveles de tensión y dispositivos de protección a utilizar en este tipo de microrredes.

Microrredes híbridas (M-CACC): combinan buses de CA y CC interconectados mediante convertidores electrónicos de potencia bidireccionales, permitiendo aprovechar las ventajas de ambos. Estos sistemas presentan flexibilidad para integrar recursos y cargas en CA y CC al mismo tiempo; hay mayor eficiencia energética (menos conversiones extras); tienen mejor capacidad de resiliencia y adaptabilidad en ambos modos de operación (autónomo y conectado a red). Pero requieren una mejor coordinación del control entre los dos buses y presentan mayor complejidad en el diseño.

En los últimos años las M-CC han experimentado un desarrollo significativo debido a que ofrecen ventajas notables en relación a eficiencia energética, simplicidad de control y adaptabilidad a las tecnologías modernas. Estas características las hacen especialmente idóneas para entornos con alta penetración de fuentes de generación renovable (como paneles fotovoltaicos) y cargas electrónicas, que son naturalmente de CC \cite{Mittal2022}. Para que las M-CC puedan operar adecuadamente en los modos autónomo y conectado a red, es necesario que sus componentes incorporen convertidores electrónicos de potencia (del tipo CC-CC y CC-CA). Estos dispositivos cumplen el rol de interfaz entre los elementos del sistema y el bus de tensión de CC que los interconecta. Para asegurar la estabilidad, la calidad de la energía y la eficiencia económica de la microrred, los convertidores mencionados deben operar bajo una estructura de control que contemple diversas estrategias, en función de la tarea específica que desempeñen dentro de la microrred. Entre las arquitecturas de control más aceptadas en la comunidad científica, se encuentra el “control jerárquico”, el cual organiza las funciones de control en distintos niveles, cada uno con objetivos definidos y operando en diferentes escalas de tiempo \cite{Shuai2018}.

En una M-CC donde los sistemas de generación y almacenamiento de energía se interconectan al bus de tensión de CC mediante convertidores CC-CC que operan con control local de corriente y tensión (control de nivel 0), el sistema de control jerárquico para esta microrred puede dividirse en tres niveles que actúan de manera coordinada y cumplen con los siguientes objetivos \cite{Adegboyega2025}:

Control Primario: opera a nivel local y es responsable de tareas tales como la regulación de voltaje del bus de CC y el reparto de la carga entre los recursos energéticos (generadores y almacenadores de energía) en proporción a la capacidad máxima de los mismos. Para implementar este nivel de control se utilizan únicamente mediciones locales (sin comunicación externa), operando en el orden de los milisegundos a segundos.

Control Secundario: permite la restauración de las desviaciones ocasionadas por el control primario (tensión del bus de CC) y mejora el reparto de carga. Generalmente utiliza enlaces de comunicación para recopilar datos y realizar ajustes a nivel del sistema. Actúa en el orden de los segundos a minutos.

Control Terciario: proporciona el nivel más avanzado de supervisión de la microrred, centrándose en la optimización (operación económica), la regulación global y la gestión de energía del sistema. También puede integrar demandas de la red eléctrica principal (por ej. venta de energía eléctrica) y coordinar múltiples microrredes interconectadas. Opera en el orden de los minutos a horas.

Según cómo se toman las decisiones de control y dónde reside la “inteligencia” del sistema \cite{AlIsmail2021}, el control jerárquico de la M-CC puede implementarse combinando estrategias de control centralizadas, descentralizadas y distribuidas \cite{Ahmad2025}. Las estrategias de control centralizadas utilizan un controlador único que decide actuar basándose en información global de la microrred; si bien esto favorece una coordinación precisa entre los componentes del sistema, el control es vulnerable a fallos en el controlador y/o las comunicaciones, pudiendo paralizar la operación de la microrred. En las estrategias de control descentralizado, cada controlador aplicado a los convertidores opera con mediciones locales, sin comunicación con otros dispositivos; esto reduce la vulnerabilidad a fallos y facilita la escalabilidad del sistema, pero limita el conocimiento del estado global del sistema, afectando a objetivos tales como el balance de potencia y/o la reducción de costos de operación de la microrred. En las estrategias de control distribuidas, los controladores de los convertidores intercambian información limitada para coordinar decisiones; si un convertidor falla, los demás mantienen operativa la microrred; de esta forma se combina tolerancia a fallos y conocimiento global de la microrred, pero requiere protocolos de comunicación robustos.

La gestión de energía que forma parte del control terciario de las M-CC consiste en una serie de estrategias de control aplicadas a la generación, el almacenamiento y las cargas (consumo) del sistema \cite{Shafiullah2022}. En el caso de la generación, la gestión de energía consiste en modificar el modo de operación de las interfaces (convertidores) que conectan a las fuentes de energía con el bus de tensión de CC de la microrred, cambiando entre los modos control de tensión de bus y seguimiento del punto de máxima potencia según el resultado del balance entre generación y consumo de energía en el sistema \cite{Alidrissi2021,Li2025}. Para el almacenamiento, la gestión de energía consiste en controlar la carga y descarga de las baterías utilizadas considerando el balance entre generación y consumo de energía; también se considera el estado de carga de estos dispositivos en la toma de decisiones sobre importar o exportar energía de la red eléctrica y para complementar la operación con dispositivos de almacenamiento de respuesta transitoria rápida (supercapacitores), que permiten afrontar variaciones bruscas en la tensión del bus de la microrred \cite{Lee2021,Panda2025}. Por otra parte, la gestión de energía del lado del consumo de la microrred está relacionada con la gestión de la carga, donde puede aplicarse un esquema de selección de cargas a abastecer, siendo estas clasificadas en cargas críticas y no críticas \cite{Elmorshedy2023}; las primeras son aquellas cargas cuya alimentación debe garantizarse en todo momento y el sistema de gestión de energía de la microrred da prioridad absoluta a su suministro, mientras que las segundas son las que pueden desconectarse o alimentarse solo cuando hay excedente de energía y la cantidad de energía almacenada lo permite (es decir, se activan únicamente después de asegurar la alimentación de las cargas críticas de la microrred).

Atendiendo a lo anterior, este trabajo se enfoca en la gestión del consumo (carga) de una M-CC utilizada para proporcionar parte de la energía eléctrica utilizada en un establecimiento educativo de la provincia de Misiones, que pertenece a las denominadas Escuelas de la Familia Agrícola (EFA). Estas instituciones se caracterizan por estar emplazadas en zonas rurales de la provincia, donde el suministro eléctrico de red presenta inconvenientes debido a fallas frecuentes en las líneas de transmisión o bien por la ausencia de infraestructura adecuada para su mantenimiento. A esto se suma la existencia de fuertes tormentas eléctricas y las dificultades en la reparación de las líneas que son derivadas del terreno accidentado y de la densa vegetación existente en las zonas rurales de la provincia. Los inconvenientes mencionados resultan en un servicio eléctrico rural con cortes prolongados y fluctuaciones en el voltaje del suministro, que afectan a las necesidades básicas como la refrigeración de alimentos, el bombeo de agua y la iluminación.

Para contribuir a la solución del inconveniente mencionado, el proyecto de investigación “Bombeo de agua con energías renovables, almacenamiento de energía y conexión a la red para pequeñas huertas rurales comunitarias: Estudio, diseño y puesta en funcionamiento”, en el que se enmarca este trabajo, propone el uso de una M-CC basada en paneles fotovoltaicos y almacenamiento con baterías, para abastecer parte de la instalación eléctrica del establecimiento. La Figura 1 muestra el esquema de esta microrred. La misma opera adecuadamente en los modos conectado y desconectado de la red eléctrica, gracias al sistema de control distribuido conformado por cinco agentes (AG, AB, AR, AS y AC) que actúan en forma autónoma y en conjunto para gestionar el modo de operación de las interfaces de potencia (rectificador, convertidores e inversores), asociadas a cada componente de la microrred. De esta forma, a través de los agentes se logra una gestión de los recursos energéticos (fuente y almacenamiento), interacción con la red eléctrica, gestión de las cargas (consumo) y detección/actuación ante fallas en los componentes de la microrred. Este sistema de supervisión multiagente constituye un sistema computacional distribuido, donde los agentes toman decisiones y actúan de forma autónoma a partir de información local e información distribuida por los demás agentes del sistema, con el fin de alcanzar objetivos comunes \cite{Stennikov2022}; en comparación con estrategias de control centralizadas y descentralizadas, esta estrategia de control distribuida combina tolerancia a fallos y conocimiento global de la microrred, lo cual mejora la resiliencia en su funcionamiento y también la escalabilidad en relación al agregado de nuevos componentes a la misma \cite{Altin2023}.
