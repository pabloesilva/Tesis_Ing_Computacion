\cleardoublepage                                 
\chapter{Problemática}

\section{Justificación}

\section{Objetivos}

\subsection{Objetivo General}
Desarrollar un sistema de gestión inteligente de cargas en una MR de fuentes renovables, principalmente de paneles fotovoltaicos, que optimice el consumo de energía priorizando cargas críticas en función de la disponibilidad energética, e incorpore una plataforma para la visualización de métricas relacionadas.

\subsection{Objetivos Específicos}

\begin{itemize}
    \item Identificar las necesidades y características clave para el monitoreo y control del consumo energético en una MR, seleccionando aquellas que resulten más relevantes y eficientes para los procesos de medición y transmisión de información.

    \item Diseñar e implementar un sistema integral de medición, adquisición y visualización de datos, que permita la comunicación efectiva entre nodos y facilite la optimización del uso de laenergía generada por fuentes renovables.

    \item Determinar las condiciones operativas de prueba tanto favorables como desfavorables parasometer al sistema. Para esto se pueden establecer diferentes perfiles de irradiación según lazona, como así diferentes consumos dependiendo de la estación del año, entre otras.

    \item Validar el correcto funcionamiento del sistema para las diferentes condiciones.

\end{itemize}

