\cleardoublepage                                  % 
\chapter{Análisis de Mercado}


El mercado global de energías renovables atraviesa un fuerte auge, impulsado por políticas de descarbonización, costos tecnológicamente cada vez más bajos y un compromiso creciente por parte de gobiernos y empresas con la sostenibilidad. El mercado de energía renovable sigue creciendo, y la energía solar representa una parte significativa del total, con gran expansión prevista hasta 2030 \cite{mordor2025renewablemarket}. Además, la demanda por sistemas fotovoltaicos sigue en aumento, donde proyectan que la energía solar generará aproximadamente 1,39 billones de kWh para 2025 a nivel mundial, con una tasa de crecimiento anual estimada del 7,4 \% entre 2025 y 2029 \cite{statista2025solarenergy}.
En paralelo, la necesidad de monitorización inteligente también está creciendo. El mercado de sistemas de monitoreo para plantas solares (PV Monitoring Systems) alcanzó valoraciones significativas y se espera que crezca entre 2025 y 2030, impulsado por la expansión de la capacidad solar, políticas favorables y el auge del IoT aplicado a energías renovables \cite{globenewswire2025pvmonitoring}.
Del lado de los medidores y monitores específicos, el mercado de “power meters” solares (medidores para instalaciones fotovoltaicas) también está proyectado a crecer. Se estima que este mercado alcanzará US\$ 6,46 mil millones para 2030, con un crecimiento anual de alrededor de 10,7 \% \cite{grandview2025solarpowermeter}.

En cuanto a Argentina, el sector de energías renovables ha mostrado un crecimiento sostenido en los últimos años, con una capacidad instalada que ha aumentado significativamente gracias a políticas de fomento como la ley nacional de Generación Distribuida 27.424 sancionada en el año 2017. La energía solar fotovoltaica ha sido una de las tecnologías más destacadas, con numerosos proyectos tanto a nivel residencial como comercial e industrial. La Secretaría de Energía argentina ha dado a conocer que la capacidad instalada en el sector pasó de 30,689 MW en 2023 a 58,996 MW en 2024 \cite{argentina2024reporteanual}.

\section{Seccionamiento del Mercado}
El sistema propuesto se orienta a un segmento de mercado conformado por empresas dedicadas a la instalación de sistemas de generación de energías renovables, especialmente sistemas de paneles solares. Estas empresas suelen ofrecer a sus clientes paquetes integrales que combinan equipamiento fotovoltaico, almacenamiento, inversores y servicios de monitoreo, por lo que representan un canal estratégico para la incorporación de tecnologías complementarias que mejoren el desempeño y la confiabilidad de las instalaciones. La inclusión de un sistema de gestión inteligente del consumo resulta especialmente relevante para este sector, ya que permite agregar valor en términos de eficiencia energética, continuidad del servicio y optimización del uso de los recursos disponibles en microrredes residenciales, rurales o industriales.

Dentro de este mercado pueden distinguirse empresas que trabajan con instalaciones residenciales, rurales y agroindustriales, como también en el sector industrial y comercial. En este sentido, el sistema se puede adaptar a estos escenarios donde los usuarios buscan reducir su dependencia de la red eléctrica y mitigar los efectos de los cortes de suministro eléctrico brindando una experiencia más estable. Además, suelen existir cargas críticas como bombas de agua, equipamiento de refrigeración o iluminación esencial. En estos entornos, la necesidad de priorizar consumos y mantener la operación ante variaciones en la disponibilidad energética vuelve especialmente valiosa una solución distribuida capaz de adaptarse en tiempo real a las condiciones de generación y almacenamiento.

El mercado objetivo comprende empresas que buscan diferenciar sus servicios mediante tecnologías que ofrezcan mayor estabilidad, eficiencia y resiliencia a las instalaciones energéticas. El sistema de gestión de consumo propuesto se posiciona como un complemento competitivo para sus paquetes comerciales, capaz de mejorar la calidad del servicio y ampliar las prestaciones de los sistemas fotovoltaicos ofrecidos a sus clientes.

\subsection{Análisis de demanda}

\subsection{Análisis de oferta}

\subsection{Análisis de precios}