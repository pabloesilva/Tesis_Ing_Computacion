\cleardoublepage                                 
\chapter{Problemática}

\section{Justificación}
La problemática se aborda considerando un aspecto crítico de los sistemas de generación distribuida y de las microrredes que, en la práctica, suele quedar relegado en las soluciones comerciales y académicas: la gestión distribuida, automática y priorizada de las cargas finales sin depender de un controlador central. Aunque existen sistemas que supervisan la generación y el consumo global, tal como se indicó en los antecedentes, son escasos los que actúan directamente sobre las cargas individuales en función de la disponibilidad energética real, y aún menos los que lo hacen de manera descentralizada, autónoma y con capacidad efectiva de actuación sobre los puntos de consumo. Este vacío es especialmente evidente cuando la oferta energética es variable y se requiere un mecanismo fiable para decidir qué dispositivos deben permanecer conectados y cuáles deben desconectarse sin comprometer la operación del sistema.

Esta necesidad adquiere especial relevancia en microrredes aisladas, donde la capacidad de generación es finita y la demanda varía en función de las necesidades. Tal como plantea \cite{rajbhandari2024enhanced}, estos sistemas se enfrentan habitualmente a escenarios de déficit de suministro, que obligan a las comunidades a afrontar apagones durante días de baja generación o en momentos críticos. En muchos entornos rurales —frecuentes en países en desarrollo— las inversiones necesarias para ampliar la generación o incorporar nuevos recursos no son viables en el corto plazo, lo que convierte a la gestión eficaz del lado de la demanda en un elemento indispensable para garantizar la continuidad de servicio. Bajo estas condiciones, el control puramente centralizado resulta insuficiente para reaccionar ante variaciones rápidas en la disponibilidad energética, por lo que se requiere incorporar mecanismos de actuación local capaces de priorizar cargas esenciales.

En este contexto, la propuesta técnica presentada busca cubrir precisamente esa limitación. El sistema permite determinar, de manera completamente automática, qué cargas se mantienen activas y cuáles deben desconectarse temporalmente cuando la energía disponible no alcanza para sostener toda la demanda. Para ello, monitoriza variables eléctricas relevantes en cada punto de consumo e incorpora una lógica de control ejecutada localmente en cada microcontrolador asociado, que toma decisiones según la prioridad asignada por el usuario. Este enfoque evita realizar modificaciones en los equipos que conforman la microrred —como inversores, rectificadores o bancos de baterías—, lo que facilita la instalación, conserva las garantías del equipamiento existente y evita incompatibilidades técnicas.

El sistema opera además con carácter modular, lo que permite aplicarlo selectivamente en cualquier punto de consumo. La cantidad de nodos puede ampliarse o reducirse sin alterar la arquitectura eléctrica general, algo especialmente valioso en entornos donde las necesidades energéticas varían con el tiempo o donde se incorporan cargas nuevas. Desde el punto de vista técnico, la propuesta constituye una adaptación innovadora de tecnologías conocidas —sensores de corriente, microcontroladores y comunicaciones inalámbricas de baja latencia— combinadas bajo una estrategia de control distribuido que atiende una necesidad aún no resuelta en la literatura: la actuación autónoma y priorizada directamente sobre las cargas finales.

En síntesis, no se busca controlar la microrred en su totalidad ni sustituir los sistemas de supervisión existentes, sino complementarlos mediante un mecanismo de gestión local del consumo. Aportando mejoras tangibles en eficiencia operativa, autonomía y sustentabilidad, al permitir que la demanda se adapte dinámicamente a la disponibilidad energética, fortaleciendo la resiliencia y el desempeño general de la microrred, especialmente en escenarios aislados o con restricciones severas de capacidad.
\section{Objetivos}

\subsection{Objetivo General}
Desarrollar un sistema de gestión inteligente de cargas en una MR de fuentes renovables, principalmente de paneles fotovoltaicos, que optimice el consumo de energía priorizando cargas críticas en función de la disponibilidad energética, e incorpore una plataforma para la visualización de métricas relacionadas.

\subsection{Objetivos Específicos}

\begin{itemize}
    \item Identificar las necesidades y características clave para el monitoreo y control del consumo energético en una MR, seleccionando aquellas que resulten más relevantes y eficientes para los procesos de medición y transmisión de información.

    \item Diseñar e implementar un sistema integral de medición, adquisición y visualización de datos, que permita la comunicación efectiva entre nodos y facilite la optimización del uso de laenergía generada por fuentes renovables.

    \item Determinar las condiciones operativas de prueba tanto favorables como desfavorables parasometer al sistema. Para esto se pueden establecer diferentes perfiles de irradiación según lazona, como así diferentes consumos dependiendo de la estación del año, entre otras.

    \item Validar el correcto funcionamiento del sistema para las diferentes condiciones.

\end{itemize}

