%%% Acrónimos %%%
%%%%%%%%%%%%%%%%%%%%

\newacronym{URL}{URL}
{Localizador Uniforme de Recursos (Uniform Resource Locator), una dirección web que especifica la ubicación de un recurso en Internet y el mecanismo para acceder a él}

\newacronym{HTTP}{HTTP}
{Protocolo de Transferencia de Hipertexto (Hypertext Transfer Protocol), utilizado para la comunicación y transferencia de datos en la World Wide Web}

\newacronym{WWW}{WWW}
{World Wide Web, sistema global de información basado en hipertexto que permite acceder y navegar por múltiples recursos en Internet}

\newacronym{P2P}{P2P}
{Red “peer-to-peer” (par a par), un modelo de intercambio directo de recursos entre usuarios sin necesidad de un servidor central}


%%% Definiciones %%%
%%%%%%%%%%%%%%%%%%%%

\newglossaryentry{metadatos}
{
    name = metadatos,
    description = {Datos estructurados que describen, explican y permiten localizar o gestionar recursos de información}
}

\newglossaryentry{compilador}
{
    name = compilador,
    description = {Programa que transforma código fuente escrito por un programador en instrucciones ejecutables por una computadora}
}

\newglossaryentry{plataforma}
{
    name = plataforma,
    description = {Infraestructura tecnológica y sociocultural que integra hardware, software, algoritmos y políticas, mediando prácticas sociales y producciones de contenido}
}

\newglossaryentry{multiplataforma}
{
    name = multiplataforma,
    description = {Software o tecnología capaz de ejecutarse e interoperar en múltiples sistemas operativos o entornos digitales}
}

\newglossaryentry{online}
{
    name = online,
    description = {Estado de conexión a una red digital, generalmente Internet, o actividad realizada a través de ella}
}

\newglossaryentry{twitter}
{
    name = Twitter,
    description = {Servicio de microblogging para la publicación breve y dinámica de mensajes}
}

\newglossaryentry{LaTeX}
{
    name = LaTeX,
    description = {Sistema de preparación de documentos basado en TeX, desarrollado por Leslie Lamport, ampliamente utilizado en el ámbito académico y científico}
}
