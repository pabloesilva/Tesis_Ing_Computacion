%%% Acrónimos %%%
%%%%%%%%%%%%%%%%%%%%

\newacronym{SGC}{SGC}
{Sistema de Gestión de Consumos}

\newacronym{AC}{AC}
{Agente de Carga}

\newacronym{NC}{NC}
{Nodo de Consumo}

\newacronym{MR}{MR}
{Microrred Eléctrica}

\newacronym{GD}{GD}
{Generación Distribuida}

\newacronym{GIDE}{GIDE}
{Grupo de Investigación y Desarrollo en Electrónica}

\newacronym{CAN}{CAN}
{Controller Area Network (Red de Área de Controlador), protocolo de comunicación serial robusto utilizado para la interconexión de agentes}

\newacronym{MQTT}{MQTT}
{Message Queuing Telemetry Transport, protocolo de mensajería ligero basado en el modelo publicación/suscripción}

\newacronym{PCB}{PCB}
{Printed Circuit Board (Placa de Circuito Impreso)}

\newacronym{RMS}{RMS}
{Root Mean Square (Valor Eficaz), medida estadística de la magnitud de una cantidad variable}

\newacronym{ADC}{ADC}
{Analog-to-Digital Converter (Conversor Analógico-Digital)}

\newacronym{IoT}{IoT}
{Internet of Things (Internet de las Cosas)}

%%% Definiciones %%%
%%%%%%%%%%%%%%%%%%%%

\newglossaryentry{microrred}
{
    name = Microrred,
    description = {Sistema eléctrico local que integra generación distribuida, almacenamiento y cargas, capaz de operar conectado a la red principal o en modo aislado (isla)}
}

\newglossaryentry{agente}
{
    name = Agente,
    description = {Entidad de hardware y software con capacidad de procesamiento autónomo que interactúa con otros agentes para alcanzar un objetivo de control común}
}

\newglossaryentry{firmware}
{
    name = Firmware,
    description = {Software específico que controla directamente el hardware de un dispositivo electrónico, ejecutándose típicamente en un microcontrolador}
}

\newglossaryentry{broker}
{
    name = Broker MQTT,
    description = {Servidor central en la arquitectura MQTT que recibe mensajes de los clientes publicadores y los distribuye a los clientes suscriptores}
}

\newglossaryentry{docker}
{
    name = Docker,
    description = {Plataforma de software que permite crear, probar e implementar aplicaciones rápidamente mediante el uso de contenedores estandarizados}
}

\newglossaryentry{espnow}
{
    name = ESP-NOW,
    description = {Protocolo de comunicación inalámbrica de baja latencia desarrollado por Espressif, que permite la transmisión directa de paquetes entre dispositivos sin necesidad de un router Wi-Fi}
}

\newglossaryentry{payload}
{
    name = Payload,
    description = {Carga útil de datos contenida dentro de un paquete de comunicación, excluyendo los encabezados y metadatos del protocolo}
}

\newglossaryentry{littleendian}
{
    name = Little Endian,
    description = {Formato de almacenamiento o transmisión de datos digitales en el cual el byte menos significativo se coloca en la posición más baja de memoria o se transmite primero}
}

\newglossaryentry{histeresis}
{
    name = Histéresis,
    description = {Tendencia de un sistema a conservar sus propiedades en ausencia del estímulo que las ha generado. En control, se utiliza para evitar oscilaciones rápidas alrededor de un punto de consigna}
}

\newglossaryentry{topico}
{
    name = Tópico,
    description = {Cadena de texto utilizada en MQTT para filtrar y enrutar mensajes. Los clientes se suscriben a tópicos específicos para recibir la información deseada}
}