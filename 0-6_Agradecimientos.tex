
\chapter{Agradecimientos} 


La realización de esta tesis fue posible gracias al acompañamiento y la orientación de quienes brindaron su tiempo, sus conocimientos y los recursos necesarios para llevar adelante este trabajo.

Se agradece especialmente a los directores Mgter. Ing. Guillermo A. Fernández y Dr. Ing. Fernando Botterón, por su guía académica y técnica, por las discusiones que enriquecieron el desarrollo del proyecto y por su constante disposición para resolver dudas y revisar los avances realizados.

Asimismo, se expresa reconocimiento al Grupo de Investigación y Desarrollo en Electrónica (GIDE) y al Departamento de Electrónica de la Facultad de Ingeniería, por proporcionar el espacio de trabajo, el equipamiento de laboratorio y el soporte técnico indispensables para la implementación y validación experimental del sistema.

Finalmente, se agradece a todos los docentes, compañeros y colaboradores que, de diversas formas, contribuyeron al proceso formativo y al entorno de trabajo en el que esta tesis pudo concretarse.