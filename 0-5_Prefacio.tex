\cleardoublepage
\chapter{Prefacio} 
%\addcontentsline{toc}{chapter}{Prefacio}
\vspace*{0.2\textheight}

% Falso epígrafe
\begin{flushright}
\Large \textit{
“Nunca se puede resolver un problema \\ en el mismo nivel en el que fue creado”
\bigbreak
Albert Einstein}
\end{flushright}

Esta tesis nace en el seno del Grupo de Investigación y Desarrollo en Electrónica (GIDE) de la Facultad de Ingeniería de la Universidad Nacional de Misiones, como respuesta a una problemática real y tangible que afecta a nuestra región. El trabajo se enmarca dentro del proyecto de investigación Código 16/11083-PDTS: “Bombeo de agua con energías renovables, almacenamiento de energía y conexión a la red para pequeñas huertas rurales comunitarias”.

Nuestra motivación principal surge de observar las dificultades que enfrentan las Escuelas de la Familia Agrícola (EFA) y los pequeños productores en zonas rurales, donde la inestabilidad del suministro eléctrico compromete servicios básicos como el bombeo de agua. Comprendimos que no bastaba con instalar paneles solares; era necesario dotar a la instalación de un "cerebro" capaz de gestionar la energía de forma eficiente y automática.

Así surge el Sistema de Gestión de Consumos (SGC) presentado en este documento. Nuestro objetivo fue desarrollar una solución tecnológica que no solo fuera funcional, sino también modular, económica y replicable. Durante el desarrollo, nos enfrentamos al desafío de integrar múltiples disciplinas: electrónica de potencia, sistemas embebidos y desarrollo de software.